% !TEX root = ../Defense_SiyangXie.tex

\chapter{Introduction}\label{chap:introduction}



\section{Motivation}\label{sec:intro-motivation}

Facility location decisions lie at the center of planning many infrastructure systems. In many practice, public agencies (e.g., governments) and private companies (e.g., retailers) both need to locate their facilities to serve spatially distributed demands/customers. For example, governments locate various public facilities, such as hospitals, schools, fire stations, to provide public services; retail companies determine the locations of their facilities including warehouses, assembly plants, stores, etc, to sell goods and provide business. The design of all such facility systems generally involves considerations of fixed investment of facility construction and transportation cost of serving demands, so as to maximize the operational efficiency and service profit of the system \citep{Daskin-2005}.


\section{Outline}\label{sec:intro-outline}

This dissertation is organized as follows. \autoref{chap:sensor} applies the reliable facility location modeling framework to sensor deployment context, in which sensors work in combinations to provide combinatorial service. The ideas of supporting stations are adopted with adjustment to represent the combinations of sensors. A mixed-integer mathematical model is formulated to determine the optimal sensor locations, the sensor combination plans, and the backup assignment decisions. Several approximation subroutines are carefully designed inside a Lagrangian relaxation framework to solve the model.  

